\section{INS Mechanization in Wander Frame}
The l-frame rotates continuously as it moves over the curved surface of the Earth because its y-axis always points toward the north (tangential to the meridian).
The rate of rotation becomes ever greater as the l-frame approaches the pole and will become infinite if the l-frame passes directly over the pole.
The rotational rate of the navigation l-frame over the Earth’s surface (known as the transport rate) is
\begin{equation}
    \omega_{el}^l = \begin{bmatrix}
        \omega_e \\
        \omega_n \\
        \omega_u
    \end{bmatrix}^l = \begin{bmatrix}
        -\dfrac{v_n}{R_M + h} \\[1em]
        \dfrac{v_e}{R_N + h} \\[1em]
        \dfrac{v_e\tan(\phi)}{R_N + h}
    \end{bmatrix}
\end{equation}
where $\omega_e$, $\omega_n$ and $\omega_u$ are the east, north and up angular velocity components.
The transport rate is the rate at which the l-frame rotates relative to the e-frame.
The angular rate vector of the wander frame with respect to the l-frame can be expressed (Jekeli 2001) as
\begin{equation}
    \omega_{wl}^l = \begin{bmatrix}
        0\\
        0\\
        \dot \alpha
    \end{bmatrix}
\end{equation}
and the angular rate of the wander frame with respect to the e-frame is
\begin{equation}
    \omega_{ew}^e = \omega_{lw}^l + \omega_{el}^l
\end{equation}

$$
\omega_{ew}^e = \begin{bmatrix}
    -\dot\phi \\
    \dot\lambda \cos(\phi) \\
    \dot\lambda \sin(\phi)
\end{bmatrix} + \begin{bmatrix}
    0\\
    0\\
    \dot \alpha
\end{bmatrix} = \begin{bmatrix}
    -\dot\phi \\
    \dot\lambda \cos(\phi) \\
    \dot\lambda \sin(\phi) + \dot \alpha
\end{bmatrix}
$$
and therefore the rotation rate of the w-frame with respect to the e-frame, resolved in the w-frame, is
\begin{equation}
    \omega_{ew}^w = R_l^w \omega_{ew}^e 
\end{equation}
where $R_l^w$ is the rotation matrix from the l-frame to the w-frame.
\begin{equation}
    R_w^l = \begin{bmatrix}
        \cos(\alpha) & -\sin(\alpha) & 0 \\
        \sin(\alpha) & \cos(\alpha) & 0 \\
        0 & 0 & 1
    \end{bmatrix}
\end{equation}
To force the third component $\omega_u$ to be zero we must ensure that
\begin{equation}
    \dot \alpha = -\dot\lambda \sin(\phi)
\end{equation}
The mechanization equations are equivalent to those of the l-frame except that
all the notations are for the w-frame rather than for the l-frame
\begin{equation}
    \begin{bmatrix}
        \dot{\bm{r}}^w \\
        \dot{\bm{v}}^w \\
        \dot{\bm{R}}^w_b
    \end{bmatrix}
    = \begin{bmatrix}
        D^{-1}\bm{v}^w \\[1em]
        R^w_b\bm{f}^b - (2\Omega_{ie}^w + \Omega_{ew}^w)\times \bm{v}^w + \bm{g}^w \\[1em]
        R^w_b(\bm{\Omega}_{ib}^b - \Omega_{iw}^b)
    \end{bmatrix}
\end{equation}